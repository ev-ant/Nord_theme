%\documentclass[aspectratio=1610]{beamer}
\documentclass[aspectratio=169]{beamer}
%\documentclass{beamer}

\usetheme{nord}
%\setbeamertemplate{background}{}
%%------------------------------------------------------------------------
%%    Specify here which title page to use (uncomment): 
%%------------------------------------------------------------------------
%% By default a presentation has a photo on title page
%%
% \settitlewhite   %Title page will be just white
 \settitlecolor      % Title page will be in corporate colors gradient
%%
%%------------------------------------------------------------------------
%%    Specify here which footline to use (uncomment): 
%%------------------------------------------------------------------------
%% By default there is "progress"-type footline that perform badly if there
%%  are more than 40 slides
%%
% \simplefootline  % only date and page of total  page number are shown
% \nofootline      % pretty straightforward

%  Color for quotes can be changed (for change - uncomment and/or modify the last parameter below)
%
%\colorlet{quotecolor}{nordgreendark}
%%------------------------------------------------------------------------
\title{A beautiful title can continue on the second line \\(using or not manual break)}
\date[\today]{Position or date placeholder \today}
\author[A.~Authorson]{Author Authorson}
\institute{\href{http://www.google.com}{Email or other information}}
%%----------------------------------------------------------------------

\begin{document}

\frame{\titlepage} 

\frame{\frametitle{Table of contents}\tableofcontents} 

%%---------------------------------------------
%% uncomment if you have long list of contents
%%
%% NB. Do not forget to comment previous frame if so
%% -------------------------------------------
% \frame{\frametitle{Table of contents}
% \begin{multicols*}{2}  % Change to 3 if it is needed
% \tableofcontents
% \end{multicols*}}
%%------------------------------------------------------
\begin{frame}{A new slide}
A new slide is created with \texttt{frame}-environment - i.e. $\backslash begin\{frame\}\{\textit{<Slide title>}\}... \backslash end\{frame\}$.
\vskip1cm
Additionally, a simple way to define \texttt{frame}-environment exists:
$\backslash frame\{\{\textit{<Slide title>}\}...\}$
 
\end{frame}

\section{Section 1} 
\frame{\frametitle{Colors} 
There are few colors in standard Nord University profile.\\ In this template they defined as:\\
\begin{center}
\color{nordblack}{\textbf{nordblack}}\\
\color{nordmarine}{\textbf{nordmarine}}\\
\color{nordblue}{\textbf{nordblue}}\\
\color{nordbluedark}{\textbf{nordbluedark}}\\
\color{nordgreen}{\textbf{nordgreen}}\\
\color{nordgreendark}{\textbf{nordgreendark}}\\
\color{nordgrey}{\textbf{nordgrey}}
\end{center}

And two contrast colors\\
\begin{center}
\color{nordred}{\textbf{nordred}}\\
\color{nordgold}{\textbf{nordgold}}
\end{center}
}

\subsection{Subsection no.1.1  }
\frame{ 
Slides could be without titles.\\
If you prefer to skip title, you should make it empty or omit \texttt{frametitle}
\vskip1.5cm
\begin{barquote}{Wise Guy}
As you see here, a nice formatting of quotes is done already. 
\end{barquote}}


\section{Section no. 2} 
\subsection{Lists I}
\frame{\frametitle{Shapes}
There are three supporting shapes in Nord University profile, which gain realisation through the unnumbered list structure:
\begin{outline}
 \1 blue circle
   \2 green triangle
     \3 grey square
 \0 A creation of the list therefore can be done via \texttt{outline{}/outline[enumerate]}- and \texttt{itemize/enumerate} environment.\\
However, \texttt{outline} option could be preferred, since it does not require inserting of nested environment (as \texttt{itemize} does).\\
\0 \textbf{This list is done with \texttt{outline}-environment. An item of level 0 produces normal text within the list.}
 \1 few other items in list are added
 \1 just to grow a list structure
    \2 even in depth
\end{outline}
}

\subsection{Lists II}
\frame{\frametitle{Numbered lists}

If numbers matter, another type of lists could be used.
\begin{outline}[enumerate]
 \1 It is an important first point;
 \1
     
 \1 And here is an important third point.
 
\end{outline}
}

\section{Section no.3} 
\subsection{Tables}
\frame{\frametitle{Tables with pause}
Tables with pause create an illusion of simple animation.
\uncover<4->{{\color{nordgreen}{\\ It is also possible to organize slide in columns,\\ And additionally to specify order of material uncovering}}}
\begin{columns}
\column{0.5\textwidth}
\begin{tabular}{c| K{2cm} |c|}
\hline
A & B & C \\ \hline
\pause 
1 & 2 & 3 \\  \hline
\pause 
A & B & C \\ \hline
\end{tabular} 
\column{0.5\textwidth}
\begin{table}
\uncover<5->{\caption{Random data}}
\begin{tabular}{c c c}
\uncover<5->{\textbf{Name} & \textit{Subject} & \textbf{Grade} \\ }
\uncover<7->{Frank & Chemistry & B \\ } 
\uncover<6->{Peter & Physics & A \\ }
\end{tabular} 
\end{table}

\end{columns}

\uncover<8->{\alert{NB:} table with \texttt{tabular}-environment is flashed left, while table with \texttt{table}-environment is centred.\\
If $\{K\{2cm\}\}$ is used, it creates a column of width 2cm with center alignment of content.
}
}

\section{Section no. 4}
\subsection{Blocs}
\frame{\frametitle{Blocs}
Blocks are nice things for highlighting. There are 3 \texttt{beamer}-class types of blocks:
\begin{block}{Block (yes, that simple)}
text to show inside the block
\end{block}

\begin{exampleblock}{Exampleblock}
text to show inside the block
\begin{itemize}
\item could be organized in lists
\centering
\item and centred if it is needed
\end{itemize}
\end{exampleblock}


\begin{alertblock}{\centering Alertblock}
Even block name could be centered.

Let's introduce formula here:
$$\int_0^\infty e^{-x^2} dx=\frac{\sqrt{\pi}}{2}$$
\end{alertblock}
}
\subsection{Unusual content}
\begin{colorframe}
For \emph{"Thank you"} slide (or slide with only picture), you can use \texttt{colorframe}-environment.
\\
It sets gradient background, removes frame titles and logos, and sets color of text to white.
\vskip1cm
Basic insights about how to organize presentation in \texttt{beamer}-class can be found \href{http://web.mit.edu/rsi/www/pdfs/beamer-tutorial.pdf}{\color{nordgold}here}.
\end{colorframe}

\end{document}